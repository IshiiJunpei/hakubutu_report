\documentclass[12pt]{jsarticle}
\usepackage[dvipdfmx]{graphicx}
\usepackage{wrapfig}
\usepackage{float}
\usepackage{otf}
\usepackage{longtable}
\usepackage{ulem}
\usepackage{ascmac}
\usepackage{url}
\setlength{\textwidth}{160truemm}      % テキスト幅: 160mm
\setlength{\fullwidth}{\textwidth}     % ページ全体の幅
\setlength{\oddsidemargin}{0mm}   % 左余白
\setlength{\topmargin}{-10mm}       % 上余白
\setlength{\textheight}{240truemm}     % テキスト高さ: 297-(30+30)=237mm
%\pagestyle{empty}
\title{秋田県藤里町施設研修事前学習資料}% 文書のタイトル
\date{\today}
\author{保健福祉課 石井淳平}              % 著者
%%%%%%%%%%%%%%%%
\begin{document}
%\maketitle
%%%%
別記 北海道博物館協会学芸職員部会博物館法改正にかかる会員意見

\section{博物館の理念・定義について}
\begin{enumerate}
\item 1951年に制定されたときの博物館法第2条第1項の定義は、改正で文章が追加されることがあっても、削られないことを望みます。私の管理する施設は博物館類似施設ですが、やはりこの定義を目標に、少ない職員や予算でこれまで運営してきましたので、これからも目標としてあってほしいものです。
\item 博物館は誰のためにあるか。現在の/同世代の人々だけでなく、未来の(将来の世代の)人々のためのものでもあることを明記する。
\item 博物館が対象とする「もの」は、施設内で保管や展示のできる「資料」だけでない。野外に広がる生態系や、地形・地質、大気・海洋など、形のないものも、博物館の調査研究や教育活動で取り扱う対象であることを明記する。
\item 博物館は、地域を理解し、伝えるために存在する(「地域」とは、町だったり、国だったり、地球だったりする)。どこか1ヶ所だけでやれば良いのではなく、各地域で収集保管や調査を行なうのは、そのためであること、地域性が(も)重要であることを明記する。
\item 博物館運営の根幹となる主たる役割・業務を明記したガイドラインを整備し、登録博物館の運営方針の基準となるようその内容を具体的に明示すること。
\item 「博物館」や「資料館」を名乗る施設の場合には、学芸員発令された職員を必置し、あるべきその職務を明示すること。
\item 昨今、観光分野への期待が大きくなりつつあるが、博物館は資料収集、資料保存、調査研究といった地道で基礎的な活動を根幹とし、こうした活動が担保されてはじめて、枝葉である情報発信、普及活動や観光利用などがはじめて展開することを改めて整理し、明記すること。順序が逆になると、博物館はやせ細るだけである。
\item 博物館の調査研究機関としての側面を強調しすぎることは、それぞれの地域の状況に即した博物館の多様な活躍の場を狭めてしまうことになりはしないか。社会教育事業に特化した博物館、自らは調査研究せずに外部研究者の調査研究をコーディネートするような学芸員の在り方があってもよいのではないか。
\item 日本の博物館全体の底上げをはかり、飽和状態にある各地の資料・標本が適切に未来へ継承されるよう、資料保存機関としての役割を十分に果たせる体制がとれるような制度設計となること。
\item 社会教育とも文化財とも異なる、「博物館行政」と「学芸員行政」を整えること。
\item 専門職としての学芸員が正しく任用され、設置主体の種類に関わらず博物館人としての職責をまっとうできるような制度設計となること。会計年度任用職員を1人配置しただけのような脆弱な体制や、兼務の学芸員資格保有者を庁内(社内)に置くことで良しとするような見せかけの学芸員配置を全廃させること。
\item 乱用される「博物館」の名称やイメージの整理をはかり、国際社会で通用するMuseumとしての位置付けを確立して、「博物館」の機能、組織、あり方について、設置主体がどのようなものでも共有できる「博物館であれば最低限備えていなければならない条件」を明確にすること。
\item 全ての博物館が、ICOMの博物館職員倫理規定にしたがって、専門職としてのある程度の独立性をもって職務にあたれるよう、法律的に明記すること。
\item 博物館法の改正にあわせ、関連法制についても迅速な整備をおこなうこと。
\item 基礎自治体に任せていても、もはや博物館は絶対に良くならないのではないか?その前提で、国全体で日本の博物館のあり方を継続的に考えていく。
\end{enumerate}

%%%
\section{学芸員資格について}
\begin{enumerate}
	\item 学芸員資格と就職の結びつきが弱く、資格取得講座の受講意欲に個人差が大きい。
	\item 学芸員資格の内容は現状維持で、博物館及び文化財を扱う職種採用時の必須要件とする。
	\item 学芸員として任用する側のメリット・インセンティブが不明瞭であり、博物館運営の金銭的負担・義務が強く感じられる。
	\item 学芸員資格が博物館の多様な実態・専門性に必ずしも対応しきれていない。
	\item 学芸員の諸活動、特に調査研究が職務上担保されていないことが多い。
	\item 既職者については猶予期間を設け、教員免許の認定講習と同様のシステムで資格取得を義務づける。
	\item 上記の通り学芸員資格を博物館職員の基礎資格として位置づけ、所謂 curator とは別物とする。このことを博物館制度とリンクすることで、年間約1万人近く生まれる有資格者の就職先を生み出す。また、全職員が博物館の専門知識をもつことが担保され、有機的にステイクホルダーが増えることで博物館活動の活性化につながる。
	\item 所謂 curator 相当職の必須資格として上級資格を新設し、博物館の専門的な職を統括し調査研究に従事する職を規定する。逆に言えば、上級職・専門職への道を選ばないというキャリア選択の道を示す。
	\item 登録博物館の職員のみが「学芸員」となるのであれば、それ以外の方も何か名乗れる称号を新設していただきたいと思います。名称はやはり「学芸員」が良いのですが、区別するのであれば「学芸士」や「博物士」でも仕方ないとは思います。
	\item 類似施設及び非博物館施設の学芸員有資格職員に対する称号資格「博物館士」等の創設は賛成。学芸員資格が業務遂行条件(採用条件)として定義される頻度は高く、博物館職員の資格以外にも社会的に活用されている。そのような職員に対する資質の客観的表明手段として必要と考える。
	\item 社会教育主事が称号として「社会教育士」を名乗ることができるようになったのに準じて、有資格者が称号として「学芸員」を名乗ることができるのは、設置母体の多様化を考えると良い対応だと感じる。もし、従来の学芸員と混乱を危惧する声が多ければ「学芸士」のように、従来の学芸員と区別する称号を用いることも手段に一つと考える。
	\item 経歴などによる「学芸員」の細分化(一級、二級、三級または専門性)などは、 並行で必要だと感じる。現行の自治体の博物館条例・規則でも「学芸員を置く」としか記載されていないところが多く、職員をスキルではなく「人工」でしかとらえていない行政の人事配置の中では、どのような学芸員が求められているか、できるだけ法的にコンクリートすべきと感じている。実際、専門および実績をある程度積んだ学芸員の後任に、いわゆる短大卒の分野違いの
	学芸員有資格者が人事異動により配置され、管理者に嘆願したところ「同じ学芸員なんだから仕方ない」と言われてしまったケースもある。
	\item 博物館への定員での配置を義務づけるとともに、任用について明記すること。
	\item 現職で博物館で働く学芸員の名簿を整備し、常に正確な統計や範囲を把握できるようにすること。
	\item ICOMの職業倫理規則のような、設置者にとらわれない「学芸員」として守るべき職業倫理を法令上も明確にすること。
	\item 地方自治法、地方教育行政法を改正し、現在学校教員に対してなされている身分保障と同じ位置付けを、博物館をはじめ社会教育専門職に対しても措置すること。
	\item 専門的な教育を実施しえない短期大学での養成課程は廃止すること。
	\item 科学研究費補助金の申請資格を付与すること。
	\item 専門職としての独立性を発揮できる身分制度とすること。
	\item 登録博物館の学芸員は、学校教員と同じく都道府県費による任用とするなど、基礎自治体の財政から独立すること。
	\item 博物館以外の現場に就く学芸員に対して、実情に応じた配慮をすること。
	\item 任用にあたって、一般の公務員採用試験とは分離し、専門職として適した採用方法がとられるよう制度を整えること(インターン制の導入、自治体の枠を越えた人事交流など)
\end{enumerate}

%%%
\section{登録制度について}
\begin{enumerate}
	\item 登録博物館・博物館相当施設・博物館類似施設の定めはあるが、登録・認証を受けるメリットが小さい。また、登録後は審査・評価が行われることなく、設置者が要件を理解しない運営・人事等をすすめても何ら罰則が無いことから、名ばかりの博物館・学芸員といった事態も否定できない。
	\item 現在の登録博物館、博物館相当施設、博物館類似施設という枠組みを見直し、博物館(museum)、資料館・展示館(gallery)、博物館類似施設とし、博物館には展示・調査研究・教育普及活動と常勤 curator の任用を義務付け、研究費・事業費を定額交付する。資料館・展示館には事業費を定額交付する。なお、今後は人口減少などの中で広域圏での博物館運営などが増加することを見込み、行政への交付ではなく施設への直接助成を前提とする。
	\item 今後、登録制度が認定制度に変わるのであれば、審査基準について、博物館職員が納得する基準ではなく、利用する・支える住民が納得できる基準を望みます。市町村立の博物館であれば、地域の誇り「おらが町には博物館がある」といった気持ちになれるようなものだと良いと思っています。
	\item 「認証」の法律的意味についてどのように議論されているのか。「認定」や「認可」、「登録」との本質的な違いをどのように見据えた上で「認証」が提案されているのか。
	\item 博物館の規模や館種、設置母体の規模による段階化された認証基準が必要と考える。認証の共通基準としては、収蔵資料の管理体制(温湿度等の記録、資料の曝露、気密環境のシールの交換)、紀要・年報類の発行状況とその品質(博物館の質を示す判断基準の多くは年報類に記載されると思われるので、それらが網羅されていること)、資料に関する調査の実施状況、博物館資料を利用した学習機会の提供が考えられる。資料に関する調査や学習機会の提供は館の規模に応じた基準が望ましい。
	\item 国内における博物館・資料館など様々な形態に及ぶ「博物館」の実態を把握してその課題を明らかにすること。また地理的条件などによって課題は一様ではないことが予想されることから、各地の博物館協会等と連携して、都道府県別にその実態の把握・分析に努めること。
	\item 分析した課題をもとに、現実的な登録館数の増加をねらいとする所有者・管理者のメリット、博物館のメリット、利用者(市民)のメリットを具体的に明示し、共有すること。特に登録館数を増やすためには、所有者・管理者のメリットとしてのハード、ソフトの多様な補助金制度の整備が望まれる。
	\item 博物館(とそこで働く学芸員)の多様な業務実態があるという現状を鑑み、広く射程をとる条文としてほしい。地域の市町村立かつ少数の学芸員で運営している博物館が置いていかれてしまうのではないかと不安を感ずる。大きな県立などの博物館と小規模博物館とが断絶しないような法律として欲しい。
	\item 博物館の多様な実態に応じて、認証・インセンティブについても多様な可能性を担保してほしい。たとえば収集/調査研究/普及(観光振興)などの各項目に応じたものなど。特に資料収集や基礎研究など地味なところこそ法律で明示するなどフォローして欲しい。
	\item 博物館設置条例や規則のなかに博物館のミッションを明示することや、学芸員の業務仕様書の作成することなどを登録博物館の登録要件に加えることで、博物館が場当たり的になんでもやらされてしまう状況を回避できないか。
	\item 登録博物館は、一度登録されると放置されてしまうため、再調査は一定期間後に必要と感じる。ただし、評価指標として「来館者数の多さ」「収益」「論文の数」といった数値目標を一律に用いるのは適切ではなく、逆に数値で評価されにくい「コレクションの傾向と収集の独自性・妥当性」、「収蔵庫の確保と展示環境の確保」、「職員の配置」といった部分が適切に保たれているかを指標とするべきと考えている。
	\item また、審査主体で博物館を評価する場合は、全て一律に同じ指標で評価するのではなく「大規模館」「中規模館」「小規模館」等により、規模により評価指標を変えてほしい。例えば、都道府県レベルの施設ならば他施設への協力・支援、専門性といった部分の比重が日常業務の中に組み込まれ、中規模館なら職員の割合に比べて展示会の企画・実施のニーズが高く業務の比重も重くなり、小規模館ならほぼ1人で管理などを行うため、管理・運営等の日常業務の比重が重くなってくる、ということが想定される。
	\item 登録博物館の対象について、登録博物館であるメリットをあまり感じない施設もあるのではないか(特に小規模館)。今まで対象外とされていた国立、大学(独立行政法人含む)等の博物館はコレクションの管理、人材の確保等の措置が喫緊の課題となっていると感じる。登録博物館とすることにより、チェック機能も含めて他機関からの相互協力を得られればいいと感じている。
	\item 「博物館の要件」を厳格に定めること。現行の登録博物館、博物館相当施設は新制度に移行すると共に、審査を厳格におこない、必要に応じて指導ができるようにする。
	\item 単なる展示施設や、公民館や図書館とはことなる「調査研究にもとづく活動」ができない施設は、この際、博物館法の体系から除外し、社会教育法や観光関連法制など、別な法体系のもとで施設の役割を果たせるようにする。
	\item 学芸員の配置を義務化すると共に、職制上の発令を求めること。
	\item 館長の要件を厳格化し、学芸員か、博物館施設での経験を積んだ、博物館に関する知見を有する者とすること。また、館長の独立性について、少なくとも現在の小中学校長と同じレベルでの独立性を確立すること。
	\item 特殊資料の所蔵について、博物館でなければ認められない資料を拡大すること(現行では種の保存法の対象生物の剥製・標本は、博物館法上の博物館に認められているが、このような対象資料を拡大すること)。
	\item 基礎自治体の枠を越えて博物館を設置できるよう、法律上の地方公共団体に特別地方公共団体を設定すること。
	\item 当館は1982年に開館し,その年の3月(1983年3月)に登録博物館とされています。それ以来、今日に至るまで登録博物館とされているわけですが、正直なところ、利点がどこにあるのか良く分かりません。近年では、北海道大学の博物館実習生を受け入れる際に記載した程度でしょうか。登録博物館であることの利点は何なのか、これが明確にならないと、登録館数の増加にはつながらないと思います。また、登録制度の意味も弱いかと思います。登録による利点、そして登録することによって何が保証されるのか明確になるべきと考えています。
\end{enumerate}

%%%
\section{登録制度と連動した振興策}
\begin{enumerate}
	\item 文化庁等の各種助成事業は安定的な博物館活動の促進という意味では使いにくい。新規性、時事を追うだけでなく基盤の強化につながる財政支援があるべき。
	\item 科学研究費制度とリンクし、基礎資格としての学芸員は奨励研究への応募を可能とし、上級資格取得者には研究者番号を付与。研究者としての地位を明確にするとともに、研究についての金銭的負担の一部を博物館運営費から切り離すことで、制度を準用するメリットを見出しやすくする。
	\item 市町村の博物館の地方交付税措置について、3月分特別交付税から普通交付税に移管することを折衝する。
	\item 登録博物館に対する補助制度が「登録施設に対するインセンティブ」と表現されていることに違和感を感じる。「審議経過報告」等も含めてWGの議論は、目的と手段が倒錯しているように感じる。博物館振興政策の手段としての補助制度を適切に運用するための法的整備として法改正が行われるものと理解しているが、登録館を増やすことが目的化しているのであれば、本末転倒だ。
	\item 登録博物館を底上げする様々な補助金制度が必要。適切な補助金制度により、登録館の質的な底上げを図るとともに、未登録館が底上げに向けて登録をめざす動きが生じてくるものと思われる。その場合、認証や認証維持の難易度は多少高くても構わないと考える。
	\item 補助制度は、認証基準と連動する必要がある。つまり、博物館が認証維持に必要な事業にかかる助成であり、次のものが必要と考える。
	\begin{enumerate}
		\item 博物館修繕事業
		\item 企画展示事業
		\item 博物館資料等の調査事業
		\item 資料保存修理事業
		\item 資料購入事業
	\end{enumerate}
	\item ほとんどの館では、調査研究費は予算措置されていない。学芸員も科研費(「奨励研究」だけでなく)を申請できるようにする。
	\item 道内の館園の多くは開基〇〇年記念して開館といったところが多いのかと思います。言わば地域のモニュメントとして誕生した博物館が20年、30年経つにつれて色褪せてきて、どうモデルチェンジをしていくか求められてきている時期なのだと思いますし、国側は観光や社会教育施設としてのモデルチェンジを期待しているのでしょうね。とはいえ調査研究があって、その成果を活かすことで観光や教育に還元されるものだと思いますので、調査研究を軽視されないような仕組みが必要だと感じます。
	\item 文化庁の現状の補助・助成事業を見てみると、観光にかかわるような事業は手厚いの対して、通常の博物館業務(調査研究・展示会等)に対してはあまり手助けをしないという姿勢を感じます。登録博物館に対して、他の皆さんの意見と同様に補助金や助成金をいただけるような、首長部局に対して登録することのメリットがはっきりと見えるものを用意していただけたらと思っています。
	\item 博物館の備品に対して、予算要求する際にいつも、「何か補助金があればねぇ・・・何かないの?」と言われます。道内の館園の多くは、築年数的にも老朽化してきたところが多いと思います。老朽化したとしても、基本不具合が生じた箇所のみ最低限の修繕といった対策しか取られていないのではないかと思います。建て替えるにしても長寿命化を計るにしても、何かしらの手助けがあればいいのになと感じています。
	\item 当館では埋蔵文化財の分布調査を国の補助を受けて行っていますが、ここ数年、申請額の7〜8割程度しか認めてもらえないことからも、財源不足なのはわかりますが、少しでも助けてもらえるような登録博物館になにかしらの優遇をされるような仕組みがあればいいなと思います。
	\item 設置者が博物館を登録する必要性を実感できる(または登録しないことの不合理性を実感できる)制度を確立すること。
		\begin{itemize}
		\item (例)博物館を保有することによる地方交付税、補助金、研究費配分など財政措置
		\item (例)収蔵庫の面積および収蔵資料点数に応じた交付税配分など財政措置
		\item (例)学芸員の配置人数に応じた交付税配分など財政措置
		\item (例)博物館の設置にともない、周辺施設の環境整備に係る補助金の創設など
		\item (例)文化財保護法にもとづく指定文化財の所蔵・管理および活用に関する博物館への補助制度の創設
		\item (例)環境アセスメントなど生物多様性や地質地理水質など、環境保全行政に対する博物館機能に対する交付金や補助金制度の創設
		\item (例)博物館資料の輸送に関する減免・割引制度の導入
		\item (例)博物館出版物の「第四種郵便 学術刊行物」への適用
	\end{itemize}
	\item 博物館を科学研究費補助金を申請できる研究機関とすること。
	\item 理科教育振興法など、現在学校に適用されている教材割引などの適用範囲を、博物館にも拡大すること。
	\item 著作権法にもとづく著作権制限行為の対象に、図書館と同じく博物館を加えること。
	
\end{enumerate}

%%%
\section{その他}
\begin{enumerate}
	\item 博物館法の条文で公立博物館の設置者として「特別地方公共団体」を明記する。
	\item 公開される文化庁の事務分掌図に「博物館」の文字を明記する。
	\item 市町村では研究職俸給表の適用は困難だが、大学の教員相当の能力と業務を担っている学芸員を、一般行政職と同等の処遇で、劣悪な研究環境の下においている現状は明らかに不合理である。有能な人材を地方博物館でも配置できるためには、(1)給与等の待遇面での改善、(2)研究者としての活動が可能な職場環境の確立のいずれもが必要である。しかし、基礎自治体によっては、どちらについても確立が困難な場合が多いと見られる。その場合、基礎自治体の枠を越えた博物館の設置と学芸員の任用が将来的には不可欠になってくるのではないか?
	\item 収蔵庫が飽和状態になっているなど、資料の保存環境の劣悪化が全国で喫緊の課題となっているが、我が国の現状から将来的に基礎自治体でこれらを管理することには、現実的に大きな困難が予想される。危機的な状況下で検討時間もなく資料が一方的に廃棄されるような最悪の事態を避けるためにも、現在のうちから地方の博物館の現状を正確に調査し、改善をはかっていくことが必要である。そうした「博物館行政」の確立が求められ、そのための「博物館の再定義」が急務であると感じる。
	\item 基礎自治体の博物館・資料館の役割と、国や都道府県の役割の分担をはっきりさせ、資料や学芸員など専門職の配置について、10年20年先を見通した対策を考える必要がある。それには、専門職である学芸員が設置母体の枠を越えて組織化され、ネットワーク化されて、議論から政策立案まで出来るような組織体制の確立が急務である。同時に、既存の日本博物館協会や都道府県博物館協会の役割が非常に大きなものになってくる。現状では博物館に関するこれらの組織の脆弱さが大きな課題であり、法改正を機にこうした業界団体、専門職組織の位置付けについて再確認する必要がある。	
\end{enumerate}

\end{document}
